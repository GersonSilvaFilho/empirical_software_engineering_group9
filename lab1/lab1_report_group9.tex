% Options for packages loaded elsewhere
\PassOptionsToPackage{unicode}{hyperref}
\PassOptionsToPackage{hyphens}{url}
%
\documentclass[
]{article}
\usepackage{lmodern}
\usepackage{amssymb,amsmath}
\usepackage{ifxetex,ifluatex}
\ifnum 0\ifxetex 1\fi\ifluatex 1\fi=0 % if pdftex
  \usepackage[T1]{fontenc}
  \usepackage[utf8]{inputenc}
  \usepackage{textcomp} % provide euro and other symbols
\else % if luatex or xetex
  \usepackage{unicode-math}
  \defaultfontfeatures{Scale=MatchLowercase}
  \defaultfontfeatures[\rmfamily]{Ligatures=TeX,Scale=1}
\fi
% Use upquote if available, for straight quotes in verbatim environments
\IfFileExists{upquote.sty}{\usepackage{upquote}}{}
\IfFileExists{microtype.sty}{% use microtype if available
  \usepackage[]{microtype}
  \UseMicrotypeSet[protrusion]{basicmath} % disable protrusion for tt fonts
}{}
\makeatletter
\@ifundefined{KOMAClassName}{% if non-KOMA class
  \IfFileExists{parskip.sty}{%
    \usepackage{parskip}
  }{% else
    \setlength{\parindent}{0pt}
    \setlength{\parskip}{6pt plus 2pt minus 1pt}}
}{% if KOMA class
  \KOMAoptions{parskip=half}}
\makeatother
\usepackage{xcolor}
\IfFileExists{xurl.sty}{\usepackage{xurl}}{} % add URL line breaks if available
\IfFileExists{bookmark.sty}{\usepackage{bookmark}}{\usepackage{hyperref}}
\hypersetup{
  pdftitle={Empirical software engineering},
  pdfauthor={Anton Lutteman; Daniel Olsson; Gerson Silva Filho; Johan Mejborn},
  hidelinks,
  pdfcreator={LaTeX via pandoc}}
\urlstyle{same} % disable monospaced font for URLs
\usepackage[margin=1in]{geometry}
\usepackage{color}
\usepackage{fancyvrb}
\newcommand{\VerbBar}{|}
\newcommand{\VERB}{\Verb[commandchars=\\\{\}]}
\DefineVerbatimEnvironment{Highlighting}{Verbatim}{commandchars=\\\{\}}
% Add ',fontsize=\small' for more characters per line
\usepackage{framed}
\definecolor{shadecolor}{RGB}{248,248,248}
\newenvironment{Shaded}{\begin{snugshade}}{\end{snugshade}}
\newcommand{\AlertTok}[1]{\textcolor[rgb]{0.94,0.16,0.16}{#1}}
\newcommand{\AnnotationTok}[1]{\textcolor[rgb]{0.56,0.35,0.01}{\textbf{\textit{#1}}}}
\newcommand{\AttributeTok}[1]{\textcolor[rgb]{0.77,0.63,0.00}{#1}}
\newcommand{\BaseNTok}[1]{\textcolor[rgb]{0.00,0.00,0.81}{#1}}
\newcommand{\BuiltInTok}[1]{#1}
\newcommand{\CharTok}[1]{\textcolor[rgb]{0.31,0.60,0.02}{#1}}
\newcommand{\CommentTok}[1]{\textcolor[rgb]{0.56,0.35,0.01}{\textit{#1}}}
\newcommand{\CommentVarTok}[1]{\textcolor[rgb]{0.56,0.35,0.01}{\textbf{\textit{#1}}}}
\newcommand{\ConstantTok}[1]{\textcolor[rgb]{0.00,0.00,0.00}{#1}}
\newcommand{\ControlFlowTok}[1]{\textcolor[rgb]{0.13,0.29,0.53}{\textbf{#1}}}
\newcommand{\DataTypeTok}[1]{\textcolor[rgb]{0.13,0.29,0.53}{#1}}
\newcommand{\DecValTok}[1]{\textcolor[rgb]{0.00,0.00,0.81}{#1}}
\newcommand{\DocumentationTok}[1]{\textcolor[rgb]{0.56,0.35,0.01}{\textbf{\textit{#1}}}}
\newcommand{\ErrorTok}[1]{\textcolor[rgb]{0.64,0.00,0.00}{\textbf{#1}}}
\newcommand{\ExtensionTok}[1]{#1}
\newcommand{\FloatTok}[1]{\textcolor[rgb]{0.00,0.00,0.81}{#1}}
\newcommand{\FunctionTok}[1]{\textcolor[rgb]{0.00,0.00,0.00}{#1}}
\newcommand{\ImportTok}[1]{#1}
\newcommand{\InformationTok}[1]{\textcolor[rgb]{0.56,0.35,0.01}{\textbf{\textit{#1}}}}
\newcommand{\KeywordTok}[1]{\textcolor[rgb]{0.13,0.29,0.53}{\textbf{#1}}}
\newcommand{\NormalTok}[1]{#1}
\newcommand{\OperatorTok}[1]{\textcolor[rgb]{0.81,0.36,0.00}{\textbf{#1}}}
\newcommand{\OtherTok}[1]{\textcolor[rgb]{0.56,0.35,0.01}{#1}}
\newcommand{\PreprocessorTok}[1]{\textcolor[rgb]{0.56,0.35,0.01}{\textit{#1}}}
\newcommand{\RegionMarkerTok}[1]{#1}
\newcommand{\SpecialCharTok}[1]{\textcolor[rgb]{0.00,0.00,0.00}{#1}}
\newcommand{\SpecialStringTok}[1]{\textcolor[rgb]{0.31,0.60,0.02}{#1}}
\newcommand{\StringTok}[1]{\textcolor[rgb]{0.31,0.60,0.02}{#1}}
\newcommand{\VariableTok}[1]{\textcolor[rgb]{0.00,0.00,0.00}{#1}}
\newcommand{\VerbatimStringTok}[1]{\textcolor[rgb]{0.31,0.60,0.02}{#1}}
\newcommand{\WarningTok}[1]{\textcolor[rgb]{0.56,0.35,0.01}{\textbf{\textit{#1}}}}
\usepackage{graphicx,grffile}
\makeatletter
\def\maxwidth{\ifdim\Gin@nat@width>\linewidth\linewidth\else\Gin@nat@width\fi}
\def\maxheight{\ifdim\Gin@nat@height>\textheight\textheight\else\Gin@nat@height\fi}
\makeatother
% Scale images if necessary, so that they will not overflow the page
% margins by default, and it is still possible to overwrite the defaults
% using explicit options in \includegraphics[width, height, ...]{}
\setkeys{Gin}{width=\maxwidth,height=\maxheight,keepaspectratio}
% Set default figure placement to htbp
\makeatletter
\def\fps@figure{htbp}
\makeatother
\setlength{\emergencystretch}{3em} % prevent overfull lines
\providecommand{\tightlist}{%
  \setlength{\itemsep}{0pt}\setlength{\parskip}{0pt}}
\setcounter{secnumdepth}{-\maxdimen} % remove section numbering

\title{Empirical software engineering}
\usepackage{etoolbox}
\makeatletter
\providecommand{\subtitle}[1]{% add subtitle to \maketitle
  \apptocmd{\@title}{\par {\large #1 \par}}{}{}
}
\makeatother
\subtitle{Lab 1: Descriptive statistics, regression and hypothesis testing\\
Group 9}
\author{Anton Lutteman \and Daniel Olsson \and Gerson Silva Filho \and Johan Mejborn}
\date{16 November 2020}

\begin{document}
\maketitle

\hypertarget{exercise-1---time-to-develop}{%
\section{Exercise 1 - Time to
Develop}\label{exercise-1---time-to-develop}}

\hypertarget{a-descriptive-data}{%
\subsection{a) Descriptive data:}\label{a-descriptive-data}}

Mean = 244.625\\
Median = 231\\
Standard deviation = 83.4672591\\
Variance = 6966.7833333\\

\hypertarget{b-what-is-being-calculated}{%
\subsection{b) What is being
calculated?}\label{b-what-is-being-calculated}}

What is being calculated is the \textbf{sample} standard deviation,
since the company has provided the time spent only for 16 features
chosen at random. The population would be if we had the time for all the
features developed.

\hypertarget{c-hypothesis}{%
\subsection{c) Hypothesis}\label{c-hypothesis}}

The hypothesis is one tailed.

\textbf{h0}: mean(Time) \textless= 225 Hours\\
\textbf{h1}: mean(Time) \textgreater{} 225 Hours

\hypertarget{d-histogram}{%
\subsection{d) Histogram}\label{d-histogram}}

\begin{Shaded}
\begin{Highlighting}[]
\NormalTok{dbin<-}\KeywordTok{data.frame}\NormalTok{(d1)}
\NormalTok{histogram_with_bin <-}\StringTok{ }\ControlFlowTok{function}\NormalTok{(bin_size) \{}
\NormalTok{  title<-}\StringTok{ }\KeywordTok{paste}\NormalTok{(}\StringTok{"Histogram with Bin size = "}\NormalTok{, bin_size)}
  \KeywordTok{ggplot}\NormalTok{(}\DataTypeTok{data =}\NormalTok{ dbin, }\KeywordTok{aes}\NormalTok{(}\DataTypeTok{x=}\NormalTok{d1))}\OperatorTok{+}
\StringTok{  }\KeywordTok{geom_histogram}\NormalTok{(}\DataTypeTok{binwidth =}\NormalTok{ bin_size)}\OperatorTok{+}
\StringTok{  }\KeywordTok{labs}\NormalTok{(}\DataTypeTok{title=}\NormalTok{title, }\DataTypeTok{x=}\StringTok{'Time to develop'}\NormalTok{, }\DataTypeTok{y=}\StringTok{'Number of occurences'}\NormalTok{)}
\NormalTok{\}}

\KeywordTok{histogram_with_bin}\NormalTok{(}\DecValTok{50}\NormalTok{)}
\KeywordTok{histogram_with_bin}\NormalTok{(}\DecValTok{75}\NormalTok{)}
\KeywordTok{histogram_with_bin}\NormalTok{(}\DecValTok{100}\NormalTok{)}
\end{Highlighting}
\end{Shaded}

\includegraphics[width=0.33\linewidth]{lab1_report_group9_files/figure-latex/histplots-1}
\includegraphics[width=0.33\linewidth]{lab1_report_group9_files/figure-latex/histplots-2}
\includegraphics[width=0.33\linewidth]{lab1_report_group9_files/figure-latex/histplots-3}

Changing the amount of bins doesn't change the perspective drastically,
but we can observe that most of the values are concentrated between 200
and 250, using the bin size of 50. When we used bigger values we would
assume that most values were between 100 and 300 which gives us much
less precision on that.

\hypertarget{e-qq-plot}{%
\subsection{e) qq-Plot}\label{e-qq-plot}}

\begin{Shaded}
\begin{Highlighting}[]
\NormalTok{car}\OperatorTok{::}\KeywordTok{qqPlot}\NormalTok{(d1,}\DataTypeTok{main =} \StringTok{'qq-Plot Time to Develop'}\NormalTok{, }\DataTypeTok{ylab=} \StringTok{'Time to develop'}\NormalTok{)}
\end{Highlighting}
\end{Shaded}

\includegraphics{lab1_report_group9_files/figure-latex/qqPlot-1.pdf}

\begin{verbatim}
## [1]  3 12
\end{verbatim}

\hypertarget{f-shapiro-wilk-test}{%
\subsection{f) Shapiro-Wilk test}\label{f-shapiro-wilk-test}}

\begin{Shaded}
\begin{Highlighting}[]
\KeywordTok{shapiro.test}\NormalTok{(d1)}
\end{Highlighting}
\end{Shaded}

\begin{verbatim}
## 
##  Shapiro-Wilk normality test
## 
## data:  d1
## W = 0.92033, p-value = 0.1708
\end{verbatim}

Observing the QQ-plot of the data we can observe that data doesn't
follow a normal distribution since there are points out of the 95\%
confidence interval that we are using in the plot.

The Shapiro-Wilk test also reinforce this non-normality of the data
since it's \emph{W} value is below 1. The p-value of 0.1708 is
considered safe for this experiment since we don't have a huge sample.

\hypertarget{g-one-sample-t-test}{%
\subsection{g) One sample T-Test}\label{g-one-sample-t-test}}

\begin{Shaded}
\begin{Highlighting}[]
\KeywordTok{t.test}\NormalTok{(d1, }\DataTypeTok{mu=}\DecValTok{225}\NormalTok{ ,}\DataTypeTok{conf.level=}\FloatTok{0.95}\NormalTok{)}
\end{Highlighting}
\end{Shaded}

\begin{verbatim}
## 
##  One Sample t-test
## 
## data:  d1
## t = 0.94049, df = 15, p-value = 0.3619
## alternative hypothesis: true mean is not equal to 225
## 95 percent confidence interval:
##  200.1484 289.1016
## sample estimates:
## mean of x 
##   244.625
\end{verbatim}

\hypertarget{exercise-2---performance}{%
\section{Exercise 2 - Performance}\label{exercise-2---performance}}

\hypertarget{a-descriptive-statistics}{%
\subsection{a) Descriptive Statistics}\label{a-descriptive-statistics}}

\begin{Shaded}
\begin{Highlighting}[]
\NormalTok{psych}\OperatorTok{::}\KeywordTok{describeBy}\NormalTok{(df2}\OperatorTok{$}\NormalTok{Time,df2}\OperatorTok{$}\NormalTok{Group)}
\end{Highlighting}
\end{Shaded}

\begin{verbatim}
## 
##  Descriptive statistics by group 
## group: timeOptimized
##    vars  n mean   sd median trimmed  mad   min   max range  skew kurtosis   se
## X1    1 10   16 0.03  16.01   16.01 0.02 15.96 16.04  0.08 -0.43    -1.19 0.01
## --------------------------------------------------------------------------------- 
## group: timeOriginal
##    vars  n  mean   sd median trimmed  mad   min   max range  skew kurtosis   se
## X1    1 10 16.02 0.03  16.02   16.02 0.04 15.96 16.05  0.09 -0.43    -1.28 0.01
\end{verbatim}

\hypertarget{b-type-of-data}{%
\subsection{b) Type of data}\label{b-type-of-data}}

\begin{Shaded}
\begin{Highlighting}[]
\KeywordTok{str}\NormalTok{(df2)}
\end{Highlighting}
\end{Shaded}

\begin{verbatim}
## tibble [20 x 2] (S3: tbl_df/tbl/data.frame)
##  $ Group: Factor w/ 2 levels "timeOptimized",..: 2 1 2 1 2 1 2 1 2 1 ...
##  $ Time : num [1:20] 16 16 16 16 16 ...
\end{verbatim}

\begin{Shaded}
\begin{Highlighting}[]
\NormalTok{df2}\OperatorTok{$}\NormalTok{Group <-}\StringTok{ }\KeywordTok{as.factor}\NormalTok{(df2}\OperatorTok{$}\NormalTok{Group)}
\NormalTok{df2}\OperatorTok{$}\NormalTok{Time <-}\StringTok{ }\KeywordTok{as.numeric}\NormalTok{(df2}\OperatorTok{$}\NormalTok{Time)}

\KeywordTok{str}\NormalTok{(df2)}
\end{Highlighting}
\end{Shaded}

\begin{verbatim}
## tibble [20 x 2] (S3: tbl_df/tbl/data.frame)
##  $ Group: Factor w/ 2 levels "timeOptimized",..: 2 1 2 1 2 1 2 1 2 1 ...
##  $ Time : num [1:20] 16 16 16 16 16 ...
\end{verbatim}

\hypertarget{c-linear-model}{%
\subsection{c) Linear Model}\label{c-linear-model}}

\begin{Shaded}
\begin{Highlighting}[]
\KeywordTok{lm}\NormalTok{(Time }\OperatorTok{~}\StringTok{ }\NormalTok{Group,df2)}
\end{Highlighting}
\end{Shaded}

\begin{verbatim}
## 
## Call:
## lm(formula = Time ~ Group, data = df2)
## 
## Coefficients:
##       (Intercept)  GrouptimeOriginal  
##             16.00               0.01
\end{verbatim}

\hypertarget{d-is-the-factori-group-statistically-significant-for-this-model}{%
\subsection{d) Is the factori ``Group'' statistically significant for
this
model?}\label{d-is-the-factori-group-statistically-significant-for-this-model}}

\begin{Shaded}
\begin{Highlighting}[]
\KeywordTok{t.test}\NormalTok{(}\DataTypeTok{formula =}\NormalTok{ Time}\OperatorTok{~}\NormalTok{Group,}
       \DataTypeTok{data=}\NormalTok{df2, }\DataTypeTok{var.equal=}\NormalTok{T)}
\end{Highlighting}
\end{Shaded}

\begin{verbatim}
## 
##  Two Sample t-test
## 
## data:  Time by Group
## t = -0.79894, df = 18, p-value = 0.4347
## alternative hypothesis: true difference in means is not equal to 0
## 95 percent confidence interval:
##  -0.03629652  0.01629652
## sample estimates:
## mean in group timeOptimized  mean in group timeOriginal 
##                      16.005                      16.015
\end{verbatim}

\end{document}
