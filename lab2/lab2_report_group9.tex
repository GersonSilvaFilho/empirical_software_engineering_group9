% Options for packages loaded elsewhere
\PassOptionsToPackage{unicode}{hyperref}
\PassOptionsToPackage{hyphens}{url}
%
\documentclass[
]{article}
\usepackage{lmodern}
\usepackage{amssymb,amsmath}
\usepackage{ifxetex,ifluatex}
\ifnum 0\ifxetex 1\fi\ifluatex 1\fi=0 % if pdftex
  \usepackage[T1]{fontenc}
  \usepackage[utf8]{inputenc}
  \usepackage{textcomp} % provide euro and other symbols
\else % if luatex or xetex
  \usepackage{unicode-math}
  \defaultfontfeatures{Scale=MatchLowercase}
  \defaultfontfeatures[\rmfamily]{Ligatures=TeX,Scale=1}
\fi
% Use upquote if available, for straight quotes in verbatim environments
\IfFileExists{upquote.sty}{\usepackage{upquote}}{}
\IfFileExists{microtype.sty}{% use microtype if available
  \usepackage[]{microtype}
  \UseMicrotypeSet[protrusion]{basicmath} % disable protrusion for tt fonts
}{}
\makeatletter
\@ifundefined{KOMAClassName}{% if non-KOMA class
  \IfFileExists{parskip.sty}{%
    \usepackage{parskip}
  }{% else
    \setlength{\parindent}{0pt}
    \setlength{\parskip}{6pt plus 2pt minus 1pt}}
}{% if KOMA class
  \KOMAoptions{parskip=half}}
\makeatother
\usepackage{xcolor}
\IfFileExists{xurl.sty}{\usepackage{xurl}}{} % add URL line breaks if available
\IfFileExists{bookmark.sty}{\usepackage{bookmark}}{\usepackage{hyperref}}
\hypersetup{
  pdftitle={Empirical software engineering},
  pdfauthor={Anton Lutteman; Daniel Olsson; Gerson Silva Filho; Johan Mejborn},
  hidelinks,
  pdfcreator={LaTeX via pandoc}}
\urlstyle{same} % disable monospaced font for URLs
\usepackage[margin=1in]{geometry}
\usepackage{color}
\usepackage{fancyvrb}
\newcommand{\VerbBar}{|}
\newcommand{\VERB}{\Verb[commandchars=\\\{\}]}
\DefineVerbatimEnvironment{Highlighting}{Verbatim}{commandchars=\\\{\}}
% Add ',fontsize=\small' for more characters per line
\usepackage{framed}
\definecolor{shadecolor}{RGB}{248,248,248}
\newenvironment{Shaded}{\begin{snugshade}}{\end{snugshade}}
\newcommand{\AlertTok}[1]{\textcolor[rgb]{0.94,0.16,0.16}{#1}}
\newcommand{\AnnotationTok}[1]{\textcolor[rgb]{0.56,0.35,0.01}{\textbf{\textit{#1}}}}
\newcommand{\AttributeTok}[1]{\textcolor[rgb]{0.77,0.63,0.00}{#1}}
\newcommand{\BaseNTok}[1]{\textcolor[rgb]{0.00,0.00,0.81}{#1}}
\newcommand{\BuiltInTok}[1]{#1}
\newcommand{\CharTok}[1]{\textcolor[rgb]{0.31,0.60,0.02}{#1}}
\newcommand{\CommentTok}[1]{\textcolor[rgb]{0.56,0.35,0.01}{\textit{#1}}}
\newcommand{\CommentVarTok}[1]{\textcolor[rgb]{0.56,0.35,0.01}{\textbf{\textit{#1}}}}
\newcommand{\ConstantTok}[1]{\textcolor[rgb]{0.00,0.00,0.00}{#1}}
\newcommand{\ControlFlowTok}[1]{\textcolor[rgb]{0.13,0.29,0.53}{\textbf{#1}}}
\newcommand{\DataTypeTok}[1]{\textcolor[rgb]{0.13,0.29,0.53}{#1}}
\newcommand{\DecValTok}[1]{\textcolor[rgb]{0.00,0.00,0.81}{#1}}
\newcommand{\DocumentationTok}[1]{\textcolor[rgb]{0.56,0.35,0.01}{\textbf{\textit{#1}}}}
\newcommand{\ErrorTok}[1]{\textcolor[rgb]{0.64,0.00,0.00}{\textbf{#1}}}
\newcommand{\ExtensionTok}[1]{#1}
\newcommand{\FloatTok}[1]{\textcolor[rgb]{0.00,0.00,0.81}{#1}}
\newcommand{\FunctionTok}[1]{\textcolor[rgb]{0.00,0.00,0.00}{#1}}
\newcommand{\ImportTok}[1]{#1}
\newcommand{\InformationTok}[1]{\textcolor[rgb]{0.56,0.35,0.01}{\textbf{\textit{#1}}}}
\newcommand{\KeywordTok}[1]{\textcolor[rgb]{0.13,0.29,0.53}{\textbf{#1}}}
\newcommand{\NormalTok}[1]{#1}
\newcommand{\OperatorTok}[1]{\textcolor[rgb]{0.81,0.36,0.00}{\textbf{#1}}}
\newcommand{\OtherTok}[1]{\textcolor[rgb]{0.56,0.35,0.01}{#1}}
\newcommand{\PreprocessorTok}[1]{\textcolor[rgb]{0.56,0.35,0.01}{\textit{#1}}}
\newcommand{\RegionMarkerTok}[1]{#1}
\newcommand{\SpecialCharTok}[1]{\textcolor[rgb]{0.00,0.00,0.00}{#1}}
\newcommand{\SpecialStringTok}[1]{\textcolor[rgb]{0.31,0.60,0.02}{#1}}
\newcommand{\StringTok}[1]{\textcolor[rgb]{0.31,0.60,0.02}{#1}}
\newcommand{\VariableTok}[1]{\textcolor[rgb]{0.00,0.00,0.00}{#1}}
\newcommand{\VerbatimStringTok}[1]{\textcolor[rgb]{0.31,0.60,0.02}{#1}}
\newcommand{\WarningTok}[1]{\textcolor[rgb]{0.56,0.35,0.01}{\textbf{\textit{#1}}}}
\usepackage{graphicx,grffile}
\makeatletter
\def\maxwidth{\ifdim\Gin@nat@width>\linewidth\linewidth\else\Gin@nat@width\fi}
\def\maxheight{\ifdim\Gin@nat@height>\textheight\textheight\else\Gin@nat@height\fi}
\makeatother
% Scale images if necessary, so that they will not overflow the page
% margins by default, and it is still possible to overwrite the defaults
% using explicit options in \includegraphics[width, height, ...]{}
\setkeys{Gin}{width=\maxwidth,height=\maxheight,keepaspectratio}
% Set default figure placement to htbp
\makeatletter
\def\fps@figure{htbp}
\makeatother
\setlength{\emergencystretch}{3em} % prevent overfull lines
\providecommand{\tightlist}{%
  \setlength{\itemsep}{0pt}\setlength{\parskip}{0pt}}
\setcounter{secnumdepth}{-\maxdimen} % remove section numbering

\title{Empirical software engineering}
\usepackage{etoolbox}
\makeatletter
\providecommand{\subtitle}[1]{% add subtitle to \maketitle
  \apptocmd{\@title}{\par {\large #1 \par}}{}{}
}
\makeatother
\subtitle{Lab 2: ANOVA\\
Group 9}
\author{Anton Lutteman \and Daniel Olsson \and Gerson Silva Filho \and Johan Mejborn}
\date{01 December 2020}

\begin{document}
\maketitle

library(stargazer) \# Exercise 1 - Time to Develop

\hypertarget{a-minimum-amount-of-users}{%
\subsection{a) Minimum amount of
users}\label{a-minimum-amount-of-users}}

\begin{Shaded}
\begin{Highlighting}[]
\KeywordTok{pwr.anova.test}\NormalTok{(}\DataTypeTok{k =} \DecValTok{5}\NormalTok{, }\DataTypeTok{n =} \OtherTok{NULL}\NormalTok{, }\DataTypeTok{f =} \FloatTok{0.08}\NormalTok{, }\DataTypeTok{sig.level =} \FloatTok{0.05}\NormalTok{, }\DataTypeTok{power =} \FloatTok{0.90}\NormalTok{)}
\end{Highlighting}
\end{Shaded}

\begin{verbatim}
## 
##      Balanced one-way analysis of variance power calculation 
## 
##               k = 5
##               n = 482.3577
##               f = 0.08
##       sig.level = 0.05
##           power = 0.9
## 
## NOTE: n is number in each group
\end{verbatim}

According to the ANOVA test the amount of users in each group is
\textbf{483} for the experiment have power 0.9. The minimum amount of
monthly users that the company must have is 2415, which is 5 (the number
of groups) times the amount of users in each group.

The effect size is inversely proportional to the number of users needed
for the test. If we accept a larger effect the number of users needed
shrinks, with this power and significance levels. On the other hand, if
we make the effect smaller we need a higher number of users to this
significance level and power.

\hypertarget{b-descriptive-statistics}{%
\subsection{b) Descriptive Statistics}\label{b-descriptive-statistics}}

\begin{Shaded}
\begin{Highlighting}[]
\NormalTok{df <-}\StringTok{ }\KeywordTok{read.csv}\NormalTok{(}\DataTypeTok{file =} \StringTok{'gotaflix-abn.csv'}\NormalTok{,}\DataTypeTok{sep =} \StringTok{","}\NormalTok{)}
\NormalTok{df}\OperatorTok{$}\NormalTok{Cover <-}\StringTok{ }\KeywordTok{as.factor}\NormalTok{(df}\OperatorTok{$}\NormalTok{Cover)}
\NormalTok{df}\OperatorTok{$}\NormalTok{Engagement <-}\StringTok{ }\KeywordTok{as.numeric}\NormalTok{(df}\OperatorTok{$}\NormalTok{Engagement)}

\NormalTok{psych}\OperatorTok{::}\KeywordTok{describeBy}\NormalTok{(df}\OperatorTok{$}\NormalTok{Engagement,}\KeywordTok{list}\NormalTok{(df}\OperatorTok{$}\NormalTok{Cover), }\DataTypeTok{mat=}\NormalTok{T)}
\end{Highlighting}
\end{Shaded}

\begin{verbatim}
##     item group1 vars   n      mean        sd    median   trimmed        mad        min       max     range
## X11    1      A    1 800 0.1603672 0.1026844 0.1565033 0.1605196 0.10165375 -0.1388049 0.5430277 0.6818325
## X12    2      B    1 800 0.1597526 0.1029644 0.1586958 0.1596599 0.10462177 -0.1653220 0.5239574 0.6892794
## X13    3      C    1 800 0.1783154 0.1053693 0.1742952 0.1783728 0.10461222 -0.1222303 0.4953971 0.6176275
## X14    4      D    1 800 0.1582659 0.1047037 0.1580229 0.1596672 0.09966692 -0.2009586 0.4392143 0.6401729
## X15    5      E    1 800 0.1698140 0.1022803 0.1682641 0.1699041 0.10205328 -0.1498057 0.4774524 0.6272581
##             skew    kurtosis          se
## X11 -0.007465712 -0.05591802 0.003630441
## X12 -0.012018272  0.07174217 0.003640342
## X13  0.027066690 -0.11965960 0.003725368
## X14 -0.164163135 -0.11158151 0.003701836
## X15 -0.041525134  0.05807999 0.003616155
\end{verbatim}

\hypertarget{c-linear-model}{%
\subsection{c) Linear Model}\label{c-linear-model}}

\begin{Shaded}
\begin{Highlighting}[]
\NormalTok{lm <-}\StringTok{ }\KeywordTok{lm}\NormalTok{(Engagement }\OperatorTok{~}\StringTok{ }\NormalTok{Cover,df)}
\end{Highlighting}
\end{Shaded}

Equation that represents the model: \begin{equation}
C = 1 -> 0.160367 + 0.017948
\end{equation}

The intercept, in this case, represents the \textbf{Cover A}.

If the the model gives only \textbf{Cover C} as 1, it means that it's
the reference for all the other covers.

\hypertarget{d-normality}{%
\subsection{d) Normality}\label{d-normality}}

\begin{Shaded}
\begin{Highlighting}[]
\NormalTok{car}\OperatorTok{::}\KeywordTok{qqPlot}\NormalTok{(Engagement }\OperatorTok{~}\StringTok{ }\NormalTok{Cover,df)}
\end{Highlighting}
\end{Shaded}

\includegraphics{lab2_report_group9_files/figure-latex/1_d_1-1.pdf}

With this plots is a bit hard to be sure about the normality, but there
are no reason to believe the opposite.

\begin{Shaded}
\begin{Highlighting}[]
\NormalTok{car}\OperatorTok{::}\KeywordTok{qqPlot}\NormalTok{(lm}\OperatorTok{$}\NormalTok{residuals)}
\end{Highlighting}
\end{Shaded}

\includegraphics{lab2_report_group9_files/figure-latex/1_d_2-1.pdf}

\begin{verbatim}
## [1]  495 1295
\end{verbatim}

Also the plot of the residuals look ok, but the data visualization is a
bit hard.

\begin{Shaded}
\begin{Highlighting}[]
\KeywordTok{shapiro.test}\NormalTok{(df}\OperatorTok{$}\NormalTok{Engagement)}
\end{Highlighting}
\end{Shaded}

\begin{verbatim}
## 
##  Shapiro-Wilk normality test
## 
## data:  df$Engagement
## W = 0.99943, p-value = 0.2713
\end{verbatim}

Using the Shapiro Wilk test we can believe that the data follows a
normal distribution sing the W value is 1 and p-value is bigger than
alpha.

\hypertarget{e-scatter-plot}{%
\subsection{e) Scatter plot}\label{e-scatter-plot}}

\begin{Shaded}
\begin{Highlighting}[]
\KeywordTok{plot}\NormalTok{(lm)}
\NormalTok{car}\OperatorTok{::}\KeywordTok{leveneTest}\NormalTok{(lm)}
\end{Highlighting}
\end{Shaded}

\begin{verbatim}
## Levene's Test for Homogeneity of Variance (center = median)
##         Df F value Pr(>F)
## group    4  0.2158 0.9297
##       3995
\end{verbatim}

\includegraphics[width=0.5\linewidth]{lab2_report_group9_files/figure-latex/1_e -1}
\includegraphics[width=0.5\linewidth]{lab2_report_group9_files/figure-latex/1_e -2}
\includegraphics[width=0.5\linewidth]{lab2_report_group9_files/figure-latex/1_e -3}
\includegraphics[width=0.5\linewidth]{lab2_report_group9_files/figure-latex/1_e -4}

We interpret the null hypothesis of the test as being if the data have
homoscedasticity. Since the P-value is 0.93, it is larger than alpha and
the null hypothesis can't be rejected.

\hypertarget{f-independence-assumption}{%
\subsection{f) Independence
Assumption}\label{f-independence-assumption}}

There is no test that can run to verify the independence of the data.
It's part of the design of the experiment and should be handled in the
collection phase.

\hypertarget{g-homoscedasticity-analysis-modified-data---check-plot-arguments}{%
\subsection{g) Homoscedasticity analysis modified data - CHECK PLOT
ARGUMENTS
\textless\textless\textless\textless\textless\textless\textless\textless\textless\textless\textless\textless\textless\textless\textless\textless\textless{}}\label{g-homoscedasticity-analysis-modified-data---check-plot-arguments}}

\begin{Shaded}
\begin{Highlighting}[]
\KeywordTok{plot}\NormalTok{(lm2) }
\NormalTok{car}\OperatorTok{::}\KeywordTok{leveneTest}\NormalTok{(lm2)}
\end{Highlighting}
\end{Shaded}

\begin{verbatim}
## Levene's Test for Homogeneity of Variance (center = median)
##         Df F value    Pr(>F)    
## group    4  358.91 < 2.2e-16 ***
##       3995                      
## ---
## Signif. codes:  0 '***' 0.001 '**' 0.01 '*' 0.05 '.' 0.1 ' ' 1
\end{verbatim}

\includegraphics[width=0.5\linewidth]{lab2_report_group9_files/figure-latex/1_g -1}
\includegraphics[width=0.5\linewidth]{lab2_report_group9_files/figure-latex/1_g -2}
\includegraphics[width=0.5\linewidth]{lab2_report_group9_files/figure-latex/1_g -3}
\includegraphics[width=0.5\linewidth]{lab2_report_group9_files/figure-latex/1_g -4}

We interpret the null hypothesis of the test as being if the data have
homoscedasticity. Since the P-value is very small, and much smaller than
alpha, the null hypothesis can be rejected and we can say with
confidence that the data \textbf{does not have homoscedasticity.}

\hypertarget{h-which-art-cover-had-a-better-engagement}{%
\subsection{h) Which art cover had a better
engagement?}\label{h-which-art-cover-had-a-better-engagement}}

\begin{Shaded}
\begin{Highlighting}[]
\KeywordTok{summary}\NormalTok{(lm)}
\end{Highlighting}
\end{Shaded}

\begin{verbatim}
## 
## Call:
## lm(formula = Engagement ~ Cover, data = df)
## 
## Residuals:
##      Min       1Q   Median       3Q      Max 
## -0.35922 -0.06652 -0.00211  0.07163  0.38266 
## 
## Coefficients:
##               Estimate Std. Error t value Pr(>|t|)    
## (Intercept)  0.1603672  0.0036631  43.779  < 2e-16 ***
## CoverB      -0.0006146  0.0051804  -0.119 0.905564    
## CoverC       0.0179482  0.0051804   3.465 0.000536 ***
## CoverD      -0.0021013  0.0051804  -0.406 0.685038    
## CoverE       0.0094468  0.0051804   1.824 0.068290 .  
## ---
## Signif. codes:  0 '***' 0.001 '**' 0.01 '*' 0.05 '.' 0.1 ' ' 1
## 
## Residual standard error: 0.1036 on 3995 degrees of freedom
## Multiple R-squared:  0.005461,   Adjusted R-squared:  0.004466 
## F-statistic: 5.484 on 4 and 3995 DF,  p-value: 0.0002114
\end{verbatim}

\begin{Shaded}
\begin{Highlighting}[]
\NormalTok{car}\OperatorTok{::}\KeywordTok{Anova}\NormalTok{(lm)}
\end{Highlighting}
\end{Shaded}

\begin{verbatim}
## Anova Table (Type II tests)
## 
## Response: Engagement
##           Sum Sq   Df F value    Pr(>F)    
## Cover      0.235    4  5.4844 0.0002114 ***
## Residuals 42.884 3995                      
## ---
## Signif. codes:  0 '***' 0.001 '**' 0.01 '*' 0.05 '.' 0.1 ' ' 1
\end{verbatim}

The model is statistically significant and we reject the hypothesis that
the mean is equal for all groups.

\begin{Shaded}
\begin{Highlighting}[]
\NormalTok{tuk <-}\StringTok{ }\KeywordTok{TukeyHSD}\NormalTok{(}\KeywordTok{aov}\NormalTok{(lm))}
\KeywordTok{plot}\NormalTok{(tuk)}
\end{Highlighting}
\end{Shaded}

\includegraphics{lab2_report_group9_files/figure-latex/1_h_2-1.pdf}

We are confident that Cover C is better than A, B and D by looking at
the plot of the Tukey test. However, we can't say with confidence that C
is better than E.

\hypertarget{exercise-2---full-factorial-experiment}{%
\section{Exercise 2 - Full Factorial
Experiment}\label{exercise-2---full-factorial-experiment}}

\hypertarget{a-experimental-groups}{%
\subsection{a) Experimental Groups}\label{a-experimental-groups}}

\hypertarget{b-linear-model-equation}{%
\subsection{b) Linear model equation}\label{b-linear-model-equation}}

\hypertarget{c-anova-assumptions}{%
\subsection{c) ANOVA assumptions}\label{c-anova-assumptions}}

\hypertarget{d-anova-table}{%
\subsection{d) ANOVA table}\label{d-anova-table}}

\begin{Shaded}
\begin{Highlighting}[]
\NormalTok{df3 <-}\StringTok{ }\KeywordTok{read.csv}\NormalTok{(}\DataTypeTok{file =} \StringTok{'gotaflix-2wayANOVA.csv'}\NormalTok{,}\DataTypeTok{sep =} \StringTok{","}\NormalTok{)}
\NormalTok{df3}\OperatorTok{$}\NormalTok{Engagement <-}\StringTok{ }\KeywordTok{as.numeric}\NormalTok{(df3}\OperatorTok{$}\NormalTok{Engagement)}
\NormalTok{df3}\OperatorTok{$}\NormalTok{Cover <-}\StringTok{ }\KeywordTok{as.factor}\NormalTok{(df3}\OperatorTok{$}\NormalTok{Cover)}
\NormalTok{df3}\OperatorTok{$}\NormalTok{Summary <-}\StringTok{ }\KeywordTok{as.factor}\NormalTok{(df3}\OperatorTok{$}\NormalTok{Summary)}
\NormalTok{lm5 <-}\StringTok{ }\KeywordTok{lm}\NormalTok{(Engagement }\OperatorTok{~}\StringTok{ }\NormalTok{Cover}\OperatorTok{*}\NormalTok{Summary, df3)}
\NormalTok{stargazer}\OperatorTok{::}\KeywordTok{stargazer}\NormalTok{(lm5, }\DataTypeTok{header =}\NormalTok{ F)}
\end{Highlighting}
\end{Shaded}

\begin{table}[!htbp] \centering 
  \caption{} 
  \label{} 
\begin{tabular}{@{\extracolsep{5pt}}lc} 
\\[-1.8ex]\hline 
\hline \\[-1.8ex] 
 & \multicolumn{1}{c}{\textit{Dependent variable:}} \\ 
\cline{2-2} 
\\[-1.8ex] & Engagement \\ 
\hline \\[-1.8ex] 
 CoverGenre & $-$0.021$^{***}$ \\ 
  & (0.006) \\ 
  & \\ 
 SummaryGenre & $-$0.019$^{***}$ \\ 
  & (0.006) \\ 
  & \\ 
 CoverGenre:SummaryGenre & 0.025$^{***}$ \\ 
  & (0.008) \\ 
  & \\ 
 Constant & 0.175$^{***}$ \\ 
  & (0.004) \\ 
  & \\ 
\hline \\[-1.8ex] 
Observations & 3,200 \\ 
R$^{2}$ & 0.005 \\ 
Adjusted R$^{2}$ & 0.004 \\ 
Residual Std. Error & 0.115 (df = 3196) \\ 
F Statistic & 5.465$^{***}$ (df = 3; 3196) \\ 
\hline 
\hline \\[-1.8ex] 
\textit{Note:}  & \multicolumn{1}{r}{$^{*}$p$<$0.1; $^{**}$p$<$0.05; $^{***}$p$<$0.01} \\ 
\end{tabular} 
\end{table}

\end{document}
